% Technological of Costa Rica 
% Operations Research
% 3rd Project
% Linear Programming
% Alonso Rivas Solano (2014079916)
% Daniel Herrera Brenes (2015130539)
% Edisson López Díaz (2013103311)
 
\documentclass{beamer} 
\usetheme[progressbar=frametitle]{metropolis} 
\setbeamertemplate{frame numbering}[fraction] 
\useoutertheme{metropolis} 
\useinnertheme{metropolis} 
\usefonttheme{metropolis} 
\usecolortheme{metropolis} 
\usepackage[utf8]{inputenc} 
\usepackage{lmodern} 
\usepackage[T1]{fontenc} 
\usepackage[spanish]{babel} 
\usepackage{tikz} 
\usepackage{natbib} 
\usepackage{hyperref} 
\usepackage{multirow} 
\usepackage{colortbl} 
\usepackage{helvet} 
\usepackage[export]{adjustbox} % loads also graphicx 
\usepackage{lipsum} 
%Definiciones 
\definecolor{color_columna_candidata}{rgb}{0, 0.424, 0.455} 
\definecolor{color_pivote}{rgb}{0.973, 0.80, 0.341} 
\definecolor{color_blanco}{rgb}{1,1,1} 
% Commands 
\newcommand\tab[1][1cm]{\hspace*{#1}}  
\newcommand\minitab[1][0.5cm]{\hspace*{#1}}  
% Tittle information 
\title{Simplex} 
\subtitle{Operations Research} 
\author[A. \& D. \& E.]{% 
\texorpdfstring{% 
\begin{columns} 
\column{.33\linewidth} 
\centering 
\\  Daniel Herrera  \\ 2015130539 \\ 
\column{.33\linewidth} 
\centering 
\\  Edisson López \\ 2013103311 \\ 
\column{.33\linewidth} 
\centering 
\\ Alonso Rivas \\ 2014079916 \\ 
\end{columns} 
} 
{Author 1, Author 2, Author 3} 
} 
\date{} 
\institute{% 
\texorpdfstring{% 
\begin{columns} 
\column{.9\linewidth} 
\centering 
\\ 
Tecnológico de Costa Rica \\ 
Semester 1, 2018 \\ 
May 24, 2018 
\end{columns} 
} 
} 
%Inicio del documento 
\begin{document} 

% - - 1st Slide - - ; 
% - - Cover - - - - ; 
\begin{frame}[plain,t] 
\maketitle 
\end{frame} 


% - - - - - - - - - ;
% - - - - 2 - - - - ;
% Algoritmo Símplex: uno o dos slides que expliquen
% un poco el algoritmo Sı́mplex.
\section{Simplex Algorithm}
\begin{frame}
The simplex is a method to solve lineal programming problems. This is a mechanical method that search for the best or optimal solution for a lineal programming(LP) problem. It was invented by George Danzig in 1947. It uses operations over a matrix to search for the optimal solution. It begin from a feasible region and it starts to do some operations, depending if you are maximizing or minimizing that search for the candidate column and the pivot, and after all the numbers are positive or negative, depends if maximizing or minimizing, that it give you the best solution.
\end{frame}

 
\section{Original Problem}  
\begin{frame}[shrink]  
\frametitle{Problema sin nombre} 
\begin{alertblock}{Maximize} 
\begin{itemize} 
\item $Z = -1 x1$ 
\end{itemize} 
\end{alertblock} 
\begin{alertblock}{Constraints} 
\begin{enumerate} 
\item $ 1x1  \leq 22$ 
\end{enumerate} 
\end{alertblock} 
\end{frame} 

\section{Initial Table} 
 
\begin{frame}  
\frametitle{Initial Table} 
\begin{table}[H] 
\begin{center} 
\resizebox{\linewidth}{!}{ 
\begin{tabular}{|*{4}{c|}} 
\hline 
\textbf{Z}  & \textbf{x1} & \textbf{s$_{1}$} & \textbf{•} \\\hline \hline 
1 & -1 & 0 & 0 \\\hline 
0 & 1 & 1 & 22\\ 
\hline 
\end{tabular}} 
\caption{Initial Table.} 
\end{center} 
\end{table} 
\end{frame} 
 
\section{Final Table} 
 
\begin{frame}  
\frametitle{Final Table} 
\begin{table}[H] 
\begin{center} 
\resizebox{\linewidth}{!}{ 
\begin{tabular}{|*{4}{c|}} 
\hline 
\textbf{Z}  & \textbf{x1} & \textbf{s$_{1}$} & \textbf{•} \\\hline \hline 
1 & 0 & 1 & 22 \\\hline 
0 & 1 & 1 & 22\\ 
\hline 
\end{tabular}} 
\caption{Final Table.} 
\end{center} 
\end{table} 
\end{frame} 
 

\section{Solution} 
\begin{frame} 
\frametitle{Solution} 
\begin{exampleblock}{Optimal solution} 
{\scriptsize Problema sin nombre} 
\begin{itemize} 
\item $Z = 22$ 
\item $x_{1} = 22$ 
\end{itemize} 
\end{exampleblock} 
\end{frame} 


\begin{frame} 
\frametitle{Especial Cases} 
The problem did not have special cases 
\end{frame} 

\begin{frame}\frametitle{}\begin{center}{\Huge - final slide -}\end{center}\end{frame} 
\end{document}
% } DOCUMENT 
% Última línea del documento